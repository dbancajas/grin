\title{Grin Developer Reference Manual}
\author{
	Dean Ancajas \\
}
\date{\today}

\documentclass[12pt]{article}

\begin{document}
\maketitle

\begin{abstract}
This document will give give details on how Grin, a minimimal implementation of the mimble wimble protocol is implemented in Rust. We will give details regarding
design decisions made  blah blah blah.
\end{abstract}

\section{Introduction}
Grin is an implementation of the MimbleWimble\ref{} protocol that is written from scratch using the Rust language\ref{}. The reason for using Rust are based on a lot memory-safe features provided by the language. As such, a lot of code listings in this document will be in Rust. \footnote{The details are outside the scope of this document but the Rust's benefits can be seen in \ref{}}

\paragraph{Outline}

We start by giving an overview of the Grin reference client in Section \ref{}. The details of the block validation are described in Section~\ref{block_validation}.
Finally, Section~\ref{conclusions} gives the conclusions.

\section{Grin Architecture Overview}

\section{Block Validation Pipeline}\label{validation}
A much longer \LaTeXe{} example was written by Gil~\cite{Gil:02}.

\section{Consensus}\label{results}
In this section we describe the results.

\section{}\label{conclusions}
We worked hard, and achieved very little.

\bibliographystyle{abbrv}
\bibliography{main}

\end{document}
This is never printed
